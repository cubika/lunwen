\newpage
\begin{center}
	\heiti\zihao{3}\textbf{海量数据实时性查询研究}
\end{center}
\begin{center}
	\heiti\zihao{-3}\textbf{摘\quad 要}
\end{center}
\vspace{2.5mm}
\songti\zihao{-4}

为了应对互联网上爆发性的信息量的存储和查询要求,目前有许多种的解决方案。传统的方法使用RDBMS关系型数据库如mysql及它的查询引擎来处理,但这需要维护非常多的表间约束,以及它不能灵活的扩展等。Hadoop是一个开源的分布式架构,在其之上的项目hive提供了功能强大的引擎来和Hadoop分布式优势相结合。HBase和MongoDB都是典型的非关系型数据库,其中HBase面向列族,MongoDB功能强大面向文档,他们都提供了独特的数据库存储模式。

本论文采用了横向比较实时性信息的方法,主要关注于查询的时间消耗,总共安排了大小数量即简单复杂查询共四种情况进行对比。经过具体的实际测试,发现mysql在中小数量级上的查询性能有明显的优势,只是在非常大的数量上的复杂查询性能不如hive.由于HBase本身不能支持条件查询,最初考虑采用hive和hbase整合的方法,但结果发现并不理想。

为了提高实时性,本论文挑选了几个非常重要的方面进行讨论。如采用索引,调整参数等。并在最后对连接查询的几个主流实现算法进行了研究。

HBase在官网里提供了api使得客户端能够通过编写java程序来操作Hbase中存在的数据。本文最后部分参考了官网的手册学习了几个重要概念并且针对前面做过的实验写出了相应的程序并成功得到了结果。

\vspace{3mm}
\zihao{-4}\heiti\textbf{关键字}\quad \songti Hive \quad HBase \quad 实时性查询