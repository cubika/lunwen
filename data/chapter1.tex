\chapter{概述}
本章主要介绍课题研究的相关背景,以及主要用到的开源软件的特点和基本概念。
\section{背景介绍}

\subsection{大量的数据}
  当前,随着整个互联网的快速发展,海量的数据每天都在无时不刻的产生。一方面,有类似facebook、twitter等社交网站如雨后春笋般的涌现,每个用户都要产生大量的动态信息,包括用户进行发布个人状态,日志等资料,以及在网站上的点击,收听音乐等操作,这些都会以日志的方式记录下来,可想而知信息量是多么大。事实上,facebook每天都要产生超过2TB的数据量,而这还是经过了科学的压缩之后得到的。除了社交网络之外,电子商务网站也需要产生大量的日志用来记录用户的行为。另一方面,从中小型到大型的数据密集型企业,如电信,金融,政府,零售等等也需要保存几十个到上百个TB的用户数据。


根据\cite{HadoopGuide}的介绍:
\begin{compactitem}
\item 纽约证券交易所每天产生1TB的交易数据
\item 著名社交网站Facebook的主机存储着约100亿张照片,占据PB级存储空间
\item 互联网档案馆存储着约2PB数据,并以每月至少20TB的速度增长
\item 瑞士日内瓦附近的大型强子对撞机每年产生约15PB的数据
这么多的数据带来的问题就是如何进行存储,如何在这些数据之上进行查询以发掘有效信息,如何在查询时保证良好的实时性。
\end{compactitem}

\subsection{传统方法存在的问题}
  在数据量很小的时候,大都是将数据部署在一个服务器上,并将数据存储在关系型数据库RDBMS中。然而,这种方法在超大规模和高并发的需求面前显得力不从心,这主要体现在:
\begin{itemize}
\item 对数据库高并发读写的需求

  web2.0网站要根据用户个性化信息来实时生成动态页面和提供动态信息,所以基本上无法使用动态页面静态化技术,因此数据库并发负载非常高,往往要达到每秒上万次读写请求。关系数据库应付上万次SQL查询还勉强顶得住,但是应付上万次SQL写数据请求,硬盘IO就已经无法承受了。其实对于普通的BBS网站,往往也存在对高并发写请求的需求,例如像JavaEye网站的实时统计在线用户状态,记录热门帖子的点击次数,投票计数等,因此这是一个相当普遍的需求。 
\item  对海量数据的高效率存储和访问的需求

  以Friendfeed为例,一个月就达到了2.5亿条用户动态,对于关系数据库来说,在一张2.5亿条记录的表里面进行SQL查询,效率是极其低下乃至不可忍受的。再例如大型web网站的用户登录系统,例如腾讯,盛大,动辄数以亿计的帐号,关系数据库也很难应付。 
\item 对数据库的高可扩展性和高可用性的需求

  在基于web的架构当中,数据库是最难进行横向扩展的,当一个应用系统的用户量和访问量与日俱增的时候,你的数据库却没有办法像web server和app server那样简单的通过添加更多的硬件和服务节点来扩展性能和负载能力。对于很多需要提供24小时不间断服务的网站来说,对数据库系统进行升级和扩展是非常痛苦的事情,往往需要停机维护和数据迁移,为什么数据库不能通过不断的添加服务器节点来实现扩展呢? 
\end{itemize}
  此外,RDBMS的优势如数据库的事务一致性、多表关联复杂查询等需求并不是很强烈,因而它的优势得到了削弱。\cite{MSU-CSE-99-39}指出了RDBMS和NOSQL之间的优劣关系。

\subsection{Hadoop架构}
  hadoop是一个能够很好解决上文提到的海量数据的存储需求。事实上,作为一个分布式处理框架,它拥有者许多可贵的优点:比如高容错性、易于扩展、能够部署在十分廉价的硬件上。此外,他还能通过维护多个副本来保证可靠性。


  Hadoop包括两个部分,其一是Hadoop分布式文件系统HDFS,它很适合哪些有大数据集的应用,并提供了对数据读写的高吞吐率。HDFS是一个master/slave的结构,就通常的部署来说,在master上只运行一个NameNode,而在每一个slave上运行一个Datanode。同时它也支持传统的层次文件组织结构,同现有的一些文件系统在操作上很类似,如创建和删除文件及文件夹等等。


\begin{figure}[!ht]
\centering
\includegraphics[scale=0.5]{photo/HDFS.jpg}
\caption{HDFS架构}
\end{figure} 


  第二个部分是MapReduce编程模型。为了进行大数据量的计算,通常采用并行计算方法。MapReduce可以简化并行计算。

\section{一些解决方案}
  为了实现大量数据的实时性查询,主要有以下几种方案:
\subsection{RDBMS}
  典型的代表就是Mysql。它的优点有:模式固定,具有ACID性质和复杂的SQL查询处理引擎,能够保证数据的一致性和完整性。但另一方面,他也有许多难以避免的缺点,主要有:在上亿条记录里进行查询效率十分低下,无法进行简单的扩展。


  为了将Mysql满足海量数据查询的需要,就必须进行扩展为关系型数据库集群。当前的思路是随着数据量的增加,使得单点的结构变为主从型的分布,这样就能在从节点进行读操作,缓解了读操作时造成的IO压力。另外,如果数据库中的表数目较多,则可以将不同的表分开存储;如果某一个表太大,则可以把表分为不同的分区同样进行分布式存储。


\tikzstyle{block}=[rectangle,draw,fill=blue!20,text width=5em,
text centered,rounded corners,minimum height=4em]
\tikzstyle{line} = [draw, very thick, color=black!50]

\begin{tikzpicture}[node distance=3cm]
\node [block] (first) {单点db};
\node [block,right of=first] (second) {主从分布};
\node [block,right of=second] (third) {纵向切割};
\node [block,right of=third] (fourth) {横向切割};
\path [line] (first) -- (second);
\path [line] (second) -- (third);
\path [line] (third) -- (fourth);
\end{tikzpicture}

\subsection{Hive}
  Hive是建立在Hadoop上的数据仓库基础构架。它提供了一系列的工具,可以用来进行数据提取转化加载(ETL),这是一种可以存储、查询和分析存储在Hadoop中的大规模数据的机制。Hive定义了简单的类SQL查询语言,称为QL,它允许熟悉SQL的用户查询数据。同时,这个语言也允许熟悉MapReduce开发者的开发自定义的mapper和reducer 来处理内建的mapper和reducer 无法完成的复杂的分析工作。\cite{dataprocess}\cite{webfenxi}介绍了Hive的特点和日志分析的应用。


  下图是hive的架构: 


\begin{figure}[!ht]
\centering
\includegraphics[scale=0.6]{photo/hive.PNG} 
\caption{Hive的架构图}
\end{figure} 


  Hive的用户接口包括CLI,Client和WUI。其中最常用的是CLI,Cli启动的时候,会同时启动一个Hive副本。Client是Hive的客户端,用户连接至HiveServer。在启动Client模式的时候,需要指出HiveServer所在节点,并且在该节点启动HiveServer。WUI是通过浏览器访问Hive。


  Hive将元数据存储在数据库中,如mysql、derby。Hive中的元数据包括表的名字,表的列和分区及其属性,表的属性(是否为外部表等),表的数据所在目录等。


  解释器、编译器、优化器完成HQL查询语句从词法分析、语法分析、编译、优化以及查询计划的生成。


生成的查询计划存储在HDFS中,并在随后有MapReduce调用执行。Hive的数据存储在HDFS中,大部分的查询由MapReduce完成。

\subsection{HBase}
  HBase ,是一个高可靠性、高性能、面向列、可伸缩的分布式存储系统,利用HBase技术可在廉价PC Server上搭建起大规模结构化存储集群。


  HBase是Google Bigtable的开源实现,类似Google Bigtable利用GFS作为其文件存储系统,HBase利用Hadoop HDFS作为其文件存储系统;Google运行MapReduce来处理Bigtable中的海量数据,HBase同样利用Hadoop MapReduce来处理HBase中的海量数据;Google Bigtable利用 Chubby作为协同服务,HBase利用Zookeeper作为对应。


  下图是HBase的架构图: 


\begin{figure}[!ht]
\centering
\includegraphics[scale=0.5]{photo/hbase.JPG}
\caption{HBase的架构图}
\end{figure}


  HBase的表由行和列组成,列划分为若干个列族。表中的row key是用来检索记录的主键,访问表中的行可以通过单个row key访问,也可以通过row key的range,还有就是全表扫描。行的一次读写是原子操作。


  Hbase表中的每一列都归属于某个列族,列族是表的chema的一部分(而列不是),必须在使用表之前定义,列名都以列族作为前缀。访问控制、磁盘和内存的使用统计都是在列族层面进行的。下面是一个表的示例:
\clearpage

\begin{figure}[!ht]
\centering
\includegraphics[]{photo/hbaseTable.PNG} 
\caption{HBase的表结构}
\end{figure} 

  从上面的表结构可以看出,Hbase的列不仅可以为空,亦可以动态进行增加,这就相对比较灵活。此外,Hbase的存储是自动分片的,少了人工操作的麻烦。它的缺点是没有一个功能强大的查询引擎,不支持复杂查询。

\subsection{MongoDB}
MongoDB是一个介于关系数据库和非关系数据库之间的产品,是非关系数据库当中功能最丰富,最像关系数据库的。他支持的数据结构非常松散,是类似json的bjson格式,因此可以存储比较复杂的数据类型。Mongo最大的特点是他支持的查询语言非常强大,其语法有点类似于面向对象的查询语言,几乎可以实现类似关系数据库单表查询的绝大部分功能,而且还支持对数据建立索引。\cite{MongoDBGuide}这本书比较详细的介绍了MongoDB的特性。

它的特点是高性能、易部署、易使用,存储数据非常方便。主要功能特性有:
\begin{compactitem}
\item 面向集合存储,易存储对象类型的数据。
\item 模式自由。
\item 支持动态查询。
\item 支持完全索引,包含内部对象。
\item 支持查询。
\item 支持复制和故障恢复。
\item 使用高效的二进制数据存储,包括大型对象(如视频等)。
\item 自动处理碎片,以支持云计算层次的扩展性。
\item 支持RUBY,PYTHON,JAVA,C++,PHP等多种语言。
\item 文件存储格式为BSON(一种JSON的扩展)。
\item 可通过网络访问。
\end{compactitem}
